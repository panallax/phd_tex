\usepackage{graphicx}
\usepackage{amsmath}
\setcounter{MaxMatrixCols}{12}
\usepackage[dvipsnames]{xcolor}
\usepackage[T1]{fontenc}
\usepackage{lmodern}
\usepackage{float}
\usepackage{epstopdf}
\usepackage{svg}
\usepackage[activate={true,nocompatibility},final,tracking=true,kerning=true,factor=1100,stretch=10,shrink=10]{microtype}
\usepackage{optidef}
\usepackage{import}
\usepackage{verbatim}
\usepackage{listings}
\usepackage{array}
\usepackage{pgfplots}
\usepackage{multirow}
\definecolor{keywordcolor}{RGB}{139,0,255}     % fuerte violeta
\definecolor{basecolor}{RGB}{0,102,204}        % azul claro
\definecolor{functioncolor}{RGB}{255,172,0}    % amarillo fuerte
\definecolor{commentcolor}{RGB}{0,180,0}       % verde brillante
\definecolor{numbercolor}{RGB}{180,180,0}      % verde amarillento
\definecolor{symbolcolor}{RGB}{0,0,0}          % negro para símbolos
\definecolor{typecolor}{RGB}{0,180,180}        % verde turquesa
\definecolor{bgcolor}{RGB}{245,248,255}

\lstdefinestyle{customdark}{
    backgroundcolor=\color{bgcolor},
    basicstyle=\ttfamily\color{basecolor}\footnotesize,
    keywordstyle=\color{keywordcolor}\bfseries,
    identifierstyle=\color{basecolor},
    commentstyle=\color{commentcolor},
    stringstyle=\color{typecolor},
    numbers=left,
    numbersep=8pt,
    stepnumber=1,
    showstringspaces=false,
     numberstyle=\tiny\color{black},
    breaklines=false,
    rulecolor=\color{black},
    frame=single,
    tabsize=4,
    captionpos=b,
    language=Python,
    morekeywords=[2]{parent},
    keywordstyle=[2]\color{basecolor},
    emph={list, str, dict, int, float, set},
    emphstyle=\color{typecolor},
    emph=[2]{find, union, append, group\_segments, values},
    emphstyle=[2]\color{functioncolor},
    literate=
      {:}{{\textcolor{symbolcolor}{:}}}1
      {=}{{\textcolor{symbolcolor}{=}}}1
      {.}{{\textcolor{symbolcolor}{.}}}1
      {(}{{\textcolor{symbolcolor}{(}}}1
      {()}{{\textcolor{symbolcolor}{()}}}1
      {)}{{\textcolor{symbolcolor}{)}}}1
      {[}{{\textcolor{symbolcolor}{[}}}1
      {]}{{\textcolor{symbolcolor}{]}}}1
      {|}{{\textcolor{symbolcolor}{|}}}1
      {,}{{\textcolor{symbolcolor}{,}}}1
      {\{}{{\textcolor{symbolcolor}{\{}}}1
      {\}}{{\textcolor{symbolcolor}{\}}}}1
      {!}{{\textcolor{symbolcolor}{!}}}1
}

\usepackage{enumitem}
\usepackage[
backend=bibtex,
style=ieee,
citestyle=numeric,
sorting=none,
maxnames=3,
minnames=1
]{biblatex}
\addbibresource{bib.bib}
\usepackage{hyperref}
\hypersetup{
    colorlinks=true,
    linkcolor=blue,
    urlcolor=blue,
    citecolor=blue,
    }
\usepackage{afterpage}
\usepackage{caption}
\usepackage{cancel}
\captionsetup{font=small,labelfont=bf}
\usepackage{subcaption}
\usepackage{tikz}
\usetikzlibrary{positioning, patterns}
\usepackage{listings}
\newcommand{\hsp}{\hspace{20pt}}
\setlength{\parskip}{1em}

\usepackage[nameinlink,noabbrev,capitalize]{cleveref}

\makeatletter
\renewcommand\part{%
  \if@openright
    \cleardoublepage
  \else
    \clearpage
  \fi
  \thispagestyle{plain}%
  \if@twocolumn
    \onecolumn
    \@tempswatrue
  \else
    \@tempswafalse
  \fi
  \null\vfil
  \secdef\@part\@spart}
\makeatother

\usepackage[
    inner=30mm, % margen interior (izquierdo en pares, derecho en impares)
    outer=25mm, % margen exterior
    top=30mm,
    bottom=25mm,
    a4paper
]{geometry}

% ---------------- FUENTE ----------------
\renewcommand{\baselinestretch}{1.5} % interlineado 1.5

% ---------------- COLOR ----------------
\usepackage{xcolor}
% Pantone Process Black = negro puro
\definecolor{PantoneBlack}{RGB}{45,41,38} % o simplemente black
\definecolor{nebrijaRed}{RGB}{186,12,47}
\color{PantoneBlack}

% ---------------- JUSTIFICACIÓN Y SANGRÍA ----------------
\usepackage{setspace}
\setstretch{1.5} % interlineado
\setlength{\parindent}{1cm} % sangría de 1 cm
%\setlength{\parskip}{0pt} % sin espacio entre párrafos
\raggedbottom
\usepackage{microtype} % mejora espaciado de texto
\usepackage{ragged2e}
\justifying % texto justificado

% ---------------- CAPÍTULOS Y TÍTULOS ----------------
\usepackage{titlesec}
\usepackage{etoolbox}
\usepackage{xcoffins}
% Evitar guiones automáticos en los títulos
\pretocmd{\chapter}{\hyphenpenalty=10000\exhyphenpenalty=10000\sloppy}{}{}

%------
\titleformat{\chapter}[display]
  {\normalfont\bfseries\color{nebrijaRed}}
  {\begin{flushright}
    \color{PantoneBlack}
     \vspace*{-6cm} % mueve el número más arriba
     \fontfamily{pzc}
     \fontsize{90}{100}\selectfont\thechapter
   \end{flushright}}
  {-5 em}
  {\Huge\bfseries\color{nebrijaRed}\MakeUppercase}
  [\vspace{-1.3em}\tikz{\draw[nebrijaRed, line width=0.8pt] (0,0)--(\textwidth,0);}]
  
% Espaciado antes y después del título
\titlespacing*{\chapter}{0pt}{0pt}{20pt}
%------
% ---
\newcommand\headerdisplay[1]{%
  \huge
  \vskip.5\baselineskip
  \filcenter\MakeUppercase{#1}%
  \vskip.0\baselineskip
}
\NewCoffin\mytmpa
\NewCoffin\mytmpb
\newcommand\placeabove[3][0pt]{%
 \SetHorizontalCoffin\mytmpa{#2}%
 \SetHorizontalCoffin\mytmpb{#3}%
 \JoinCoffins*\mytmpb[hc,t]\mytmpa[hc,b](0pt,#1)%
 \TypesetCoffin\mytmpb
}

\renewcommand\thepart{\Roman{part}}
\titleclass{\part}{top} % make part like a chapter
\titleformat{\part}[frame]
  {\normalfont \bfseries\color{nebrijaRed}}
  {\filcenter\placeabove[2\baselineskip]
  {\fontsize{42}{50}\selectfont\color{nebrijaRed}$\mathcal{PART}$}{\fontfamily{pzc}\selectfont\fontsize{90}{100}\selectfont\bfseries\hspace{0.2em}\color{PantoneBlack}\thepart\hspace{0.2em}}}
  {0pt}
  {\headerdisplay}
\titlespacing*{\part}{0pt}{0.3\textheight}{40pt}


\titleformat{\section}
  {\normalfont\Large\bfseries\color{nebrijaRed}}
  {\thesection}{1em}{}

\titleformat{\subsection}
  {\normalfont\large\bfseries\color{nebrijaRed}}
  {\thesubsection}{1em}{}

% ---------------- CABECERAS Y PIE DE PÁGINA ----------------
\usepackage{fancyhdr}
\fancyhf{}

\fancyfoot[R]{\thepage}

% Encabezado:
% - En páginas pares (izquierda): título del capítulo
% - En páginas impares (derecha): título de la sección
\fancyhead[LE]{\large\nouppercase{\leftmark}}
\fancyhead[RO]{\large\nouppercase{\rightmark}}

% Línea de encabezado
\renewcommand{\headrulewidth}{0.4pt}
\renewcommand{\footrulewidth}{0pt}

% Aplica el estilo
\pagestyle{fancy}

% --- Estilo limpio para capítulos ---
\fancypagestyle{plain}{
  \fancyhf{}
  \fancyfoot[LE,RO]{\thepage}
  \renewcommand{\headrulewidth}{0pt}
}

% --- Asegura que los capítulos usen el estilo 'plain' ---
\let\oldchapter\chapter
\renewcommand{\chapter}{%
  \clearpage%
  \thispagestyle{plain}%
  \oldchapter%
}

% ---------------- CAPÍTULOS EN PÁGINA DERECHA ----------------
\usepackage{etoolbox}
\makeatletter
\pretocmd{\chapter}{\cleardoublepage}{}{}
\makeatother

% ---------------- PDF ----------------
\pdfminorversion=4
\pdfcompresslevel=9
\pdfobjcompresslevel=3


\let\cleardoublepage\clearpage
\usepackage{subfiles}