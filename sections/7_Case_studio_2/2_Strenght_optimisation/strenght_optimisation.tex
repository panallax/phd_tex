\documentclass[../../../main.tex]{subfiles}
\begin{document}

The algorithm developed and explained in the methodology section of this manuscript is used to create self-supporting lattice structures. 
To do this, it is ensured that all edges meet strict geometric requirements. 
The geometry obtained may not have the distribution of nodes and edges that allows for a better distribution of stresses within the structure. 
It may be that a very wide distribution of path lengths is obtained through which stresses travel from the upper base to the lower base. 
This would cause some edges to be much more loaded than others. 
Modifying the position of the nodes to homogenise the length of these paths would ensure that the stresses are also evenly distributed throughout the structure.
This would prevent certain areas from easily exceeding critical stresses. 
The problem with this proposal is that it requires the modification of the position of the nodes and edges, causing the structure to lose its self-supporting condition. 
Therefore, once the geometry of the structure is defined, the only remaining variable that can contribute to rigidity to the structure is the section of the bars. 
Knowing that the moment of inertia of a circular bar depends on the elastic modulus of the material, which will be constant, and the radius of the section, increasing the section implies increasing the moment of inertia. 
In addition, increasing the moment of inertia increases the critical Euler buckling load of the bar. 
The greater the critical Euler load, the greater the compressive strength of the bar. 
However, it must be taken into account that the greater the radius of the section, the greater the mass of the bar. 
The greater the mass of the bar, the lower its first natural frequency. 
This is counterproductive for vibration testing. 
If the objective is to optimise a structure against vibration, the aim is for its first natural frequency to be higher than the test frequency.
Increasing the mass of the structure causes this to decrease. 
Therefore, the overall objective of the optimisation should be to increase the radius of the bars but decrease the total mass. 
To do this, it is proposed to find the combination of sections for all bars that provides the highest compressive strength and the lowest possible mass. 
To achieve this, it will be necessary to obtain the stress field along the structure and see which bars are more loaded and which are less. 
The most stressed bars will need larger radii, but the less stressed bars may have their cross-section reduced. 
Taking advantage of the fact that the structure is reticular, it was decided to use matrix analysis of structures to create a multi-objective optimisation problem. 
Although matrix analysis of structures does not offer real results, because of the simplifications assumed, it allows stress at the nodes to be calculated very quickly. 
Therefore, its inclusion in an iterative optimisation process is feasible. 
Another alternative would be to perform the calculations using finite element analysis (FEA) simulations, but this would greatly increase the computing time, and the results might not be so different from those obtained using matrix analysis.   
This section will explain how the matrix analysis was developed and its integration into the optimisation algorithm used.

\subsection{Formulation of the global stiffness matrix}


Matrix analysis of structures allows nodal displacements and internal forces in complex structural systems to be determined systematically using an algebraic approach. The main objective is to solve the matrix system:

\begin{equation}\label{eq:stiff}
\mathbf{K}_\text{global} \, \mathbf{u} = \mathbf{F},
\end{equation}

where $\mathbf{K}_\text{global}$  is the global stiffness matrix, $\mathbf{u}$ is the nodal displacement vector, and $\mathbf{F}$ is the vector of applied external forces. 
The resolution of this system allows the displacements at all nodes to be obtained. 
From these displacements, the internal forces in each member of the structure can be obtained. 
Using these forces, the nodal stress field will be obtained, and the state of each of the bars will be evaluated.
To do this, the global stiffness matrix of the system and the nodal force vector must first be obtained. 
In this case, only vertical forces applied to the upper nodes will be considered.

Each bar has been modelled as a three-dimensional bar element capable of withstanding tension, compression, bending and torsion. 
Circular-section bars have been used as they are the simplest to manufacture subsequently. 
For this kind of bar, the local stiffness matrix in local coordinates $(x',y',z')$ is expressed as

\begin{equation}
\mathbf{k}_e^\text{local} =
\begin{bmatrix}
\mathbf{K}_{ii} & \mathbf{K}_{ij} \\
\mathbf{K}_{ji} & \mathbf{K}_{jj}
\end{bmatrix},
\end{equation}

where the blocks $\mathbf{K}_{pq}$ ($p,q = i,j$) represent the interactions between nodes i and j, and each node has three degrees of freedom for displacement and three additional degrees of freedom for rotation
($u_x, u_y, u_z, \theta_x, \theta_y, \theta_z$). 
The blocks are defined as:

\begin{align}
\mathbf{K}_{ii} &= 
\begin{bmatrix}
\frac{EA}{L} & 0 & 0 & 0 & 0 & 0 \\
0 & \frac{12EI_z}{L^3} & 0 & 0 & 0 & \frac{6EI_z}{L^2} \\
0 & 0 & \frac{12EI_y}{L^3} & 0 & -\frac{6EI_y}{L^2} & 0 \\
0 & 0 & 0 & \frac{GJ}{L} & 0 & 0 \\
0 & 0 & -\frac{6EI_y}{L^2} & 0 & \frac{4EI_y}{L} & 0 \\
0 & \frac{6EI_z}{L^2} & 0 & 0 & 0 & \frac{4EI_z}{L}
\end{bmatrix}, \\
\mathbf{K}_{ij} &= 
\begin{bmatrix}
-\frac{EA}{L} & 0 & 0 & 0 & 0 & 0 \\
0 & -\frac{12EI_z}{L^3} & 0 & 0 & 0 & \frac{-6EI_z}{L^2} \\
0 & 0 & -\frac{12EI_y}{L^3} & 0 & \frac{6EI_y}{L^2} & 0 \\
0 & 0 & 0 & -\frac{GJ}{L} & 0 & 0 \\
0 & 0 & \frac{6EI_y}{L^2} & 0 & \frac{2EI_y}{L} & 0 \\
0 & -\frac{6EI_z}{L^2} & 0 & 0 & 0 & \frac{2EI_z}{L}
\end{bmatrix}
\end{align}


where $E$ is the modulus of elasticity, $A = \pi r^2$ is the area of the circular cross-section, $I_y$ and $I_z$ are the moments of inertia, $G$ is the shear modulus, $J$ is the polar moment of inertia, and $L$ is the length of the bar.
Furthermore, as they are symmetric matrices, it follows that $\mathbf{K}_{ji} = \mathbf{K}_{ij}^\mathsf{T}  \;\text{and}\;
\mathbf{K}_{jj} = \mathbf{K}_{ii}.$


This matrix is represented with respect to the local coordinate system, which has its origin at one end of the bar and one of the three axes coincides with the axis of the bar. 
In order to incorporate this matrix into the global matrix, it must be transformed into a global coordinate system shared by all bars. 
To do this, a transformation matrix, $\mathbf{T}$, is calculated.
This matrix transforms this local matrix into global coordinates so that 

\begin{equation}
\mathbf{k}_e^\text{global} = \mathbf{T}^\mathsf{T} \, \mathbf{k}_e^\text{local} \, \mathbf{T}.
\end{equation}

This matrix is constructed based on the cosines of the direction vectors of each bar, which represent the orientation of the local axes with respect to the global system.
The vector of the longitudinal axis of the element is obtained from the coordinates of its nodes $i$ and $j$:
\begin{equation}
\vec{\mathbf{x}} = \frac{\mathbf{r}_j - \mathbf{r}_i}{L}, 
\quad L = \|\mathbf{r}_j - \mathbf{r}_i\|,
\end {equation}
where $\mathbf{r}_i=(x_i,y_i,z_i)$ and $\mathbf{r}_j=(x_j,y_j,z_j)$ are the coordinates of the nodes. This unit vector defines the direction of the local axis $\vec{\mathbf{x}}$.
Next, the local vectors $\vec{\mathbf{y}}$ and $\vec{\mathbf{z}}$ are constructed, forming a right-handed orthonormal system. 
In the general case, an auxiliary vector is taken in the $XZ$ plane and projected perpendicularly onto $\mathbf{x}'$. The other vectors are then calculated as follows:
    \begin{equation}
    \vec{\mathbf{z}} = \frac{\vec{\mathbf{x}} \times \mathbf{aux}}{\|\vec{\mathbf{x}} \times \mathbf{aux}\|}, 
    \qquad
    \vec{\mathbf{y}} = \vec{\mathbf{z}} \times \vec{\mathbf{x}}.
    \end {equation}
This procedure ensures that $(\vec{\mathbf{x}},\vec{\mathbf{y}},\vec{\mathbf{z}})$ form an orthonormal basis.
Once the vectors defining the local basis have been obtained, their transformation to a global basis is obtained using the direction cosines of each of these vectors.
The direction cosines are organised in the matrix $\mathbf{R}$, where each row contains the global components of a local axis:
\begin{equation}
\mathbf{R} =
\begin{bmatrix}
x_X & x_Y & x_Z \\
y_X & y_Y & y_Z \\
z_X & z_Y & z_Z
\end{bmatrix}.
\end{equation}

Thus, $(x_X, x_Y, x_Z)$ are the direction cosines of the axis $\vec{\mathbf{x}}$, $(y_X, y_Y, y_Z)$ those of the axis $\vec{\mathbf{y}}$, and $(z_X, z_Y, z_Z)$ those of the axis $\vec{\mathbf{z}}$ with respect to the global system.
These are obtained by dividing each component by the square root of the sum of the squares of its components.
Finally, the transformation matrix of the element, of dimension $12\times 12$, is constructed by placing $\mathbf{R}$ in diagonal blocks:

\begin{equation}
\mathbf{T} =
\begin{bmatrix}
\mathbf{R} & \mathbf{0} & \mathbf{0} & \mathbf{0} \\
\mathbf{0} & \mathbf{R} & \mathbf{0} & \mathbf{0} \\
\mathbf{0} & \mathbf{0} & \mathbf{R} & \mathbf{0} \\
\mathbf{0} & \mathbf{0} & \mathbf{0} & \mathbf{R}
\end{bmatrix}.
\end{equation}

Once the local stiffness matrices $\mathbf{k}_e^\text{global}$ of each element $e$ have been obtained, they must be combined into a single global stiffness matrix $\mathbf{K}_\text{global}$ that describes the complete behaviour of the structure. 
Each node of the structure has several degrees of freedom (translational and rotational). 
For a spatial bar, each node has $6$ degrees of freedom, so that an element with two nodes has $12$. 
The local matrix $\mathbf{k}_e^\text{global}$ is defined in the subspace of these $12$ degrees of freedom, but in the global system, these degrees of freedom must be located in the positions corresponding to nodes $i$ and $j$ within the total displacement vector $\mathbf{u}$.
To do this, a location vector is defined that associates the local degrees of freedom of the element with the indices of the global matrix. 
For example, if node $i$ corresponds to degrees of freedom $[1,\dots,6]$ and node $j$ to $[7,\dots,12]$ in the global structure, then the 12 local degrees of freedom of the element are assigned to those positions.
The assembly is performed by adding the contributions of each element in the corresponding positions of the global matrix. Formally, for each element $e$:

\begin{equation}
\mathbf{K}_\text{global}[I_p, I_q] \; += \; \mathbf{k}_e^\text{global}[p,q],
\end{equation}

where $I_p$ and $I_q$ are the global indices associated with the local degrees of freedom $p$ and $q$. 
This procedure ensures that, when two or more elements share a node, their rigidities accumulate in the same position in the global matrix, reflecting the compatibility of displacements and the balance of forces. 
At the end of the assembly process for all elements, a matrix $\mathbf{K}_\text{global}$ of dimension $n_\text{dof} \times n_\text{dof}$ is obtained, where $n_\text{dof}$ is the total number of degrees of freedom of the structure.

Once the global stiffness matrix of the structure has been obtained, the nodal force vector must be defined to solve \cref{eq:stiff} and thus obtain the nodal displacement vector.
The vector $\mathbf{F}$ is constructed by assigning, for each degree of freedom of each node, the external force applied in the corresponding direction.  
To simulate a static compression case, a concentrated downward vertical load of magnitude $P$ is applied to the upper nodes. This means that the component corresponding to the degree of freedom of translation in $z$ of each upper node takes the value $F_z = -P$, while the other components remain at zero.
In this way, the vector $\mathbf{F}$ has non-zero entries only at the positions associated with the vertical displacements of the upper nodes.

In addition, the base nodes are considered as fixed supports and, therefore, their displacement and rotation are restricted.
The restrictions on the supports are imposed directly on the displacement vector $\mathbf{u}$. For the nodes located on the ground, $\mathbf{u} = 0$ is set in all their degrees of freedom, which is equivalent to cancelling both displacements and rotations. This allows the size of the system to be reduced since all the rows and columns of the global matrix associated with the restricted degrees of freedom are cancelled.
The reduced system is expressed as

\begin{equation}
\mathbf{K}_\text{red} \, \mathbf{u}_\text{free} = \mathbf{F}_\text{red},
\end{equation}

where $\mathbf{u}_\text{free}$ contains only the unrestricted degrees of freedom.  
This linear system can be solved directly, obtaining the displacements of all free nodes, from which the internal forces in the bars are subsequently calculated.

Once the global system has been solved, the global displacement vector $\mathbf{u}$ is available. 
From this, the deformations and internal forces in each bar of the structure can be calculated.
Each element $e$ connects two nodes, so its local displacement vector $\mathbf{u}_e^\text{local}$ is obtained by extracting the $12$ global degrees of freedom corresponding to the element and transforming them using the transformation matrix $\mathbf{T}_e$:
\begin{equation}
\mathbf{u}_e^\text{local} = \mathbf{T}_e \, \mathbf{u}_e^\text{global}.
\end{equation}

The relationship between displacements and internal forces in the local system can be related by means of the stiffness matrix of each element as
\begin{equation}
\mathbf{f}_e^\text{local} = \mathbf{k}_e^\text{local} \, \mathbf{u}_e^\text{local}.
\end{equation}
This vector of local forces of an element contains the stresses and moments to which it is subjected at each of its nodes
$$
\mathbf{f}_e^{\text {local }}=\left[\begin{array}{llllllllllll}
F_{x, i} & F_{y, i} & F_{z, i} & M_{x, i} & M_{y, i} & M_{z, i} & F_{x, j} & F_{y, j} & F_{z, j} & M_{x, j} & M_{y, j} & M_{z, j}
\end{array}\right]^T.
$$
In an ideal bar, internal actions appear as pairs of equal and opposite forces and moments at the two nodes, reflecting the equilibrium of the element. Therefore, the vector of internal forces of an element can be expressed as
\begin{equation}
\mathbf{f}_e^\text{internal} =
\begin{bmatrix}
N & V_y & V_z & T & M_y & M_z
\end{bmatrix}^T,
\end{equation}
where $N$ corresponds to the axial stress, $V_y$ and $V_z$ to the shear stresses, $T$ to the torsional moment, and $M_y$ and $M_z$ to the bending moments.
These values are used to evaluate the stress state of the structure, verifying the tensile, compressive, and torsional strength limits of each bar.
The matrix analysis that allows obtaining the stress field in the structural elements was performed using the Spyffness Python library, developed by the author during the course of this project.
This library allows the construction of a three-dimensional frame from a graph and performs its structural analysis.
Once the overall stiffness matrix of the structure and the corresponding stress vectors have been obtained, the results are used to optimise the sections of each of the structural elements using an optimisation algorithm, which will be explained below.

\subsection{Definition and implementation of the optimisation algorithm}

The optimisation of lattice structures is a highly non-linear, multimodal and discrete problem, where the aim is to find the combination of bar sections that provides the greatest strength with the lowest possible weight. 
The difficulty lies in the fact that the search space is very large and the structural constraints are non-linear. 
Due to these characteristics, traditional gradient-based algorithms are not very suitable, as they can get stuck in local optima or be limited by the discrete nature of the design variables. 
Therefore, genetic algorithms (GA) were used in this case. 
These algorithms are part of the evolutionary metaheuristic family and are inspired by the natural selection of biological processes, where the best generations prevail. 
Therefore, they are of interest in problems where the search space is complex, as they can escape local minima. 
Other metaheuristic alternatives, such as particle swarm algorithms (PSO), could also be a good solution to this problem, but they tend to be less efficient in multi-objective optimisations.

The algorithm used is NSGA-II (Non-dominated Sorting Genetic Algorithm II) and was implemented using the Python DEAP library\cite{DEAP_JMLR2012}. 
This algorithm aims to find a set of Pareto optimal solutions for each objective. 
To do this, it generates an initial population of \textit{N} individuals. 
In this case, each individual is a vector of length equal to the number of bars in the structure, containing the value of the section of each one. 
To reduce the non-linearity of the problem, the possible values of the diameters of each of the bars are limited according to their mechanical slenderness and their critical buckling load. 
This prevents the bars from buckling and entering a non-linear elastic regime. 
To apply Euler's buckling theory, the bars must be very slender. 
A minimum slenderness of 50 was established, so the minimum diameter that a bar can have is given by
\begin{equation}
    D_{\text{min}, \text{ slender}} = L/\lambda = L/50.
\end{equation}
To prevent the diameters from being too small, it was decided to include another criterion for selecting the minimum diameter, the buckling criterion. 
The diameter of the bar is calculated for which the bar has a critical Euler load equal to the load applied to each of the loaded nodes of the structure. 
For a circular bar, the resulting expression for the minimum diameter is as follows
\begin{equation}
D_{\text {min}, \text { bucking }}=\left(\frac{64 P_{crit }\left(K L^2\right)}{\pi^3 E}\right)^{1 / 4}.
\end{equation}
Where \textit{K} is the effective length coefficient, which is assumed to be unitary when considering that the bars are articulated. 
Once both minimum diameters have been obtained, the larger of the two is considered the minimum diameter for each bar. 
The maximum diameter is obtained by multiplying this minimum diameter by an arbitrary proportionality constant, which was set at 2.5. 
Having obtained both values, a random value is selected within the permissible range for each of the bars to generate each of the initial individuals.

An initial population of 50 individuals was established. 
These are evaluated in each of the two functions to be optimised. 
The first of these is the weight function. 
This function calculates the total weight of the structure for a set of diameters.
The second function evaluates the load state of the structure. 
To do this, three dimensionless values were used, which were called the utilisation factor ($\eta$), the buckling utilisation factor ($\eta_N$) and the bending utilisation factor ($\eta_T)$. 
The first relates the equivalent Von Mises stress and the elastic limit of the material, the second relates the nodal load to the Euler critical load, and the third relates the bending stress and the elastic limit of the material. 
To calculate each of these, the Von Mises stress is first calculated as 
\begin{equation}
    \sigma_{VM} = \sqrt{\sigma^2 + 3\tau^3}.
\end{equation}
Where $\sigma$ is the normal stress, which is obtained as $\sigma = \sigma_N + \sigma_M$.
Where $\sigma_N$ is the normal stress that is calculated as $\sigma = N/A$ if $N > 0$ or $\sigma = N/A\cdot\frac{1}{1-N/P_{crit}}$ if the bar is subjected to compression, $N<0$. 
Where $N$ is the axial force at the node, obtained from the matrix analysis, and $\frac{1}{1-N/P_{crit}}$ is an amplification factor reflecting the loss of stiffness as the load approaches buckling.
And $\sigma_M$ is the bending stress that is calculated as $\sigma_M = (M_y + M_z)r/I$.
Where $r$ is the radius of gyration and $I$ is the inertia momentum.
Then, $\tau$ is the total shear stress that is calculated as
\begin{equation}
    \tau = \sqrt{\tau^2_V + \tau^2_T}
\end{equation}
Where $\tau_V$ is the shear stress calculated as $\tau_V = \sqrt{V_y^2 + V_z^2}/(0.9A)$ considering only the 90\% of the cross-section as effective area.
And $\tau_T$ is the torsional stress calculated as $\tau_T = Tr/I$.
$V_y, V_z\text{ and }T$ are also obtained from the structural matrix analysis.
Once the Von Mises stress is obtained, it is divided by the yield stress of the chosen material to calculate the utilisation factor of the current bar.
If $\eta >1$ means that the bar has collapsed, while $\eta \leq 1$ means that the bar remains in the elastic regime and is still functional.
In practice, bars that have buckled still provide some rigidity to the structure, so a structure was allowed to have 20\% of its bars collapsed before the entire structure was considered to have collapsed. 
If more than 20\% of the sections had a utilisation factor greater than one, a penalty equivalent to ten times the difference between the number of collapsed bars and the maximum number of collapsed bars allowed was applied to that individual.

Once the total weight and maximum utilisation factor have been obtained, considering the possible penalty for each individual, the solutions obtained are classified according to their dominance. 
According to this classification, the next generation of offspring is generated by crossing and mutating the current parents. 
First, the five best solutions from each generation go directly to the second generation. 
This prevents good results obtained during mutation from being lost. 
Then, pairs of the remaining parents are generated. 
For each of them, a 70\% probability of crossing was established. 
When two parents cross, they exchange a subgroup of values between them. 
To do this, a pair of indices is randomly generated, and the values between those indices of the first parent are exchanged with the values between the same indices in the second parent. 
This is done with the intention of introducing variability in the offspring while maintaining a relationship with previous generations. 
Once crossed, a 20\% probability of mutation was established for the offspring. 
This means that if an offspring is chosen to mutate, all values will be replaced by random values of acceptable diameters. 
This is done with the intention of introducing diversity into the population and avoiding possible stagnation at a local minimum. 
After making the relevant mutations, the new generation is evaluated again by both functions, and this process is repeated until the maximum number of iterations is reached or until the weight of the structure varies by less than 0.1\% in the last five iterations. 
At that point, the optimisation is considered to have reached a global minimum.

As an example of how the optimisation algorithm works, the distribution of the diameters of the first layer of tetrahedra in a structure generated from a cylinder with a diameter of 80 \textit{mm} has been optimised. 
\cref{fig:optim} \textcolor{blue}{A} shows the load state. A point force of 1000 \textit{N} has been applied to each of the upper nodes, and the base nodes have been fixed. 
\cref{fig:optim} \textcolor{blue}{B} shows the evolution of the structure's weight, the maximum utilisation factor among the structure and the average diameter. Conventional steel has been used as the structural material, with the following mechanical properties: $E$ = 200 $GPa$, $G$= 77 $GPa$, $\nu$ = 0.3, $\rho$ = 7.85$\times10^{-6}\;kg/mm^3$ and $\sigma_Y$ = 235 $MPa$.

\begin{figure}[!htbp]
    \centering
    \begin{subfigure}[b]{0.5\textwidth}
        \includegraphics[width = \textwidth]{imgs/G_4.pdf}
        \caption{Load state used in the example structure. The total load applied to the upper nodes is 1000 \textit{N}.}
     \end{subfigure}
     \vspace{3em}
    \begin{subfigure}[b]{0.8\textwidth}
        \includegraphics[width =\textwidth]{imgs/opt.png}
        \caption{Evolution of the different variables to be optimised during optimisation process.}
     \end{subfigure}
     \caption{Result of the optimisation of the first layer of tetrahedra generated from a cylinder with a diameter of 80 \textit{mm}.}
    \label{fig:optim}
\end{figure}

Once the set of sections that improves the structure's strength and reduces its weight has been found, the next step is to generate the mesh of the final structure in order to simulate it and verify that the design effectively meets the structural requirements. 
In addition, in order to manufacture the structure using additive manufacturing, it will be necessary to have the surface mesh of the structure.

\end{document}