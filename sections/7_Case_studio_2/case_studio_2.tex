\documentclass[../../main.tex]{subfiles}

\begin{document}

The second objective of this project is to verify whether the structures developed can be used in scenarios other than bone scaffolding. 
The previous objective sought to analyse the topological characteristics of the structures to verify whether they could be printed without supports and in a non-planar form. 
This second case study aims to analyse the structures from a structural point of view. 
Taking advantage of the benefits offered by random lattice structures, it is proposed to use them as a structural element in a spacecraft. 
Specifically, the proposal is to design a structure that could serve as a bracket for the propulsion system of one of the observation satellites of the European Space Agency's (ESA) Copernicus Hyperspectral Imaging Mission for the Environment (CHIME).  

Up to this point, the design of the structures has focused on purely geometric parameters to ensure printability without support. 
If the structures are to be used as structural elements, it must be ensured that they have the required strength. 
Therefore, a strength optimisation module must be implemented for the structures. 
This chapter will explain the boundary conditions that define the problem, the logic implemented to optimise the strength of the structures, and their manufacture. 



\section{Definition of the study case and boundary conditions}
\subfile{1_Boundary_conditions/boundary_conditions}

\section{Structural optimisation focused on stiffness}
\subfile{2_Strenght_optimisation/strenght_optimisation}

\section{Structural mesh generation and analysis}
\subfile{3_Meshing/meshing}

\section{Manufacture and experimental validation of samples}
\subfile{4_Manufacturing/manufacturing}

\end{document}
