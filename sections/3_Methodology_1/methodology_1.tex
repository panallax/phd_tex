\documentclass[../../main.tex]{subfiles}

\begin{document}

In 1932, mathematician Horace Lamb said in a speech to the \textit{British Association for the Advancement of Science}:

\begin{centering}

\textit{"I am an old man now, and when I die and go to heaven, there are two matters on which I hope for enlightenment. One is quantum electrodynamics, and the other is the turbulent motion of fluids. And about the former I am rather optimistic."}
\end{centering}

Lamb was one of the pioneers of mathematical hydrodynamics and made great contributions to the field of turbulence. However, due to the highly non-linear, multi-scale and, above all, chaotic and random behaviour of turbulence, Lamb, like many other physicists, considered it unthinkable to gain a deep understanding of the phenomenon.

Randomness is something that cannot or should not be relied upon when designing physical systems. 
Sometimes, randomness is intrinsic to our system, and one must deal with it. 
But it should never be the cornerstone of our system. 
When faced with the task of designing random lattice structures, it is known in advance that randomness will be intrinsic to our problem. 
Therefore, one must decide which part of the dichotomy one wants to take: do I want randomness to dominate my system, or do I want to dominate randomness in my system? 
This chapter will show the advantages and disadvantages of letting randomness dominate our system. 
To do this, we will explain the proposed methodology for generating random lattice structures that can guarantee all the desired geometric conditions.

\section{Random points connection}
\subfile{1_Random_point_connection/random_points_connection}

\section{Pseudo-random points connection}
\subfile{2_Pseudorandom_points_generation/pseudorandom_points_generation}

\section{Controlled random connections}
\subfile{3_Controlled_random_connections/controlled_random_connections}

\end{document}