\documentclass[../../main.tex]{subfiles}

\begin{document}

Throughout this manuscript, the complete development of the algorithm for generating self-sustaining stochastic porous structures has been explained. 
The initial ideas developed to achieve the sought objective have been presented, along with evidence that these solutions were not valid. 
Subsequently, the final state reached by distilling these initial ideas is described in detail. 
All the components that form part of this algorithm are explained in the Methodology section of this manuscript. 
However, at no point is the potential of this algorithm for generating stochastic structures revealed, nor are its limitations disclosed. 
This section aims to provide an overview of all the modules that make up the package developed throughout this project. 
In addition, a study is included on how the behaviour of the structures obtained varies for different values of the design parameters.

\section{Mesh-gen: Python package modules and their integration}
\subfile{1_analysis/analysis}

\section{Capabilities and limitations of the algorithm}
\subfile{2_limitations/limitations}

\section{Quasi-static mechanical testing of the structures}
\subfile{3_testing/testing}

\end{document}
