\documentclass[../../../main.tex]{subfiles}
\begin{document}


Four designs were tested to characterise the developed structures mechanically and to compare them with the most commonly used designs in the literature: gyroid and Voronoi patterns\cite{alex}.  
For this purpose, two reference structures were generated using the built-in routines of nTop: a regular structure based on a gyroidal pattern, and a stochastic structure derived from a Voronoi tessellation. 
In addition, two structures were produced using the algorithm proposed in this work, each generated with the two mesh generation modules used in the developed package.
An example of all designs is shown in \cref{fig:mech_test}. 
Cylinders of 20x30 \textit{mm}, with a porosity of 82\%, were printed using SLA and all specimens were cured with standard UV light for 20 minutes with constant rotation.  
The mechanical behaviour of this type of structure was characterised by compression testing using a Metrotec 300K universal testing machine. 
The test conditions were carried out according to ISO 844, which applies to polymeric cells. 
A load cell of 10 \textit{kN} and a constant displacement rate of 3 \textit{mm/min} were used. 

\begin{figure}[!htbp]
    \centering
    \begin{subfigure}[t]{0.2\textwidth}
        \includegraphics[width = \textwidth]{imgs/Gyroid.pdf}
        \caption{Gyroid.}
     \end{subfigure}
     \hspace{0.3cm}
    \begin{subfigure}[t]{0.2\textwidth}
        \includegraphics[width =\textwidth]{imgs/Voronoi.pdf}
        \caption{Voronoi.}
     \end{subfigure}
     \hspace{0.3cm}
     \begin{subfigure}[t]{0.2\textwidth}
        \includegraphics[width =\textwidth]{imgs/random_cyl.pdf}
        \caption{Random with cylinders.}
     \end{subfigure}
     \hspace{0.3cm}
     \begin{subfigure}[t]{0.2\textwidth}
        \includegraphics[width =\textwidth]{imgs/random.pdf}
        \caption{Random with unique mesh.}
     \end{subfigure}
     \caption{Example of the different structures tested. a) Gyroid, b) Voronoi, c) Cylindrical random and d) Random with unique mesh.}
    \label{fig:mech_test}
\end{figure}

Five of each of the different designs of porous structures were tested to a deformation of 20\%. 
A safety criterion was used, which stopped the test when a drop in strength greater than the security margin was detected, indicating catastrophic failure of the specimen. 
\cref{fig:plot} shows the average results of each of the tests for each case. 
The curves that do not reach the 20\% deformation indicate that, in at least one of the tests, the specimen suffered a catastrophic failure, triggering the stop criterion. 
Therefore, the average values represented are the smallest deformation achieved in each case. 

\begin{figure}[!htbp]
    \centering
    \includegraphics[width= 0.9\textwidth]{imgs/plot.pdf}
    \caption{Stress-strain mean curves for all specimens.}
    \label{fig:plot}
\end{figure}

\cref{tab:results} shows the results of the different mechanical properties obtained for each of the designs. 
As these are porous structures, the elastic modulus was calculated using the slope between 0.5\% and 1\% deformation, according to ISO 844. 

\begin{table}[!htbp]
\centering
\caption{Mechanical properties obtained from the quasi-static test for each design.}
\label{tab:results}
\renewcommand{\arraystretch}{1.3}
\resizebox{\textwidth}{!}{%
\begin{tabular}{ccccc}
\hline
\textbf{Mechanical properties}     & \textbf{Gyroid}         & \textbf{Voronoi} & \textbf{Random}         & \textbf{Random with mesh} \\ \hline
\textbf{Elastic Modulus [MPa]}     & 17.23 $\pm$ 2.23          & 8.81 $\pm$ 3.4     & \textbf{20.27 $\pm$ 3.56} & 18.04 $\pm$ 0.79            \\
\textbf{Max Stress [MPa]}          & \textbf{0.835 $\pm$ 0.11} & 0.695 $\pm$ 0.03   & 0.790 $\pm$ 0.03          & 0.601 $\pm$ 0.04            \\
\textbf{Stress @ 5\% strain [MPa]}  & 0.698 $\pm$ 0.10          & 0.529 $\pm$ 0.01   & \textbf{0.706 $\pm$ 0.03} & 0.597 $\pm$ 0.05           \\
\textbf{Stress @ 10\% strain [MPa]} & \textbf{0.713 $\pm$ 0.09} & 0.565 $\pm$ 0.05   & 0.312 $\pm$ 0.3           & 0.061 $\pm$ 0.02            \\
\textbf{Absorbed Energy [J]}       & \textbf{0.105 $\pm$ 0.01} & 0.076 $\pm$ 0.003  & 0.026 $\pm$ 0.01          & 0.033 $\pm$ 0.001           \\ \hline
\end{tabular}%
}
\end{table}

The highest elastic modulus obtained after the compression tests was obtained by the stochastic structures proposed in this work.
Although the structure created by joining cylinders has shown better behaviour in compression than the one obtained using the George W. Hart algorithm. 
This algorithm tends to reinforce the nodes but, at the same time, makes the edges thinner, facilitating buckling. 
The tested structures had very long edges and high porosity. 
Thus, the proposed structures have collapsed catastrophically due to a propensity to buckle as mentioned above.
It is therefore normal that the structure made from cylinders had a better compression behaviour. 
Concerning the Gyroid and Voronoi, the proposed structures have a higher elastic modulus than both of them. 
However, they have a worse behaviour at failure, as they suffered a very fragile failure, causing the safety system of the testing machine to trigger in some cases. 
Despite this, it is remarkable that the fact of having obtained a higher modulus than the Gyroid, since it is the regular structure that presents excellent mechanical properties.
It is also noteworthy to have obtained a higher mechanical resistance than the Voronoi structure, since it is the most common structure among the random ones. 
The better behaviour after rupture of this structure compared to those proposed may be because this structure has shorter edges and, therefore, is less prone to buckling. 
However, the random orientation of the edges may explain the lower mechanical strength. 
In the proposed structures, the edges tend to have an inclination of the minimum angle established. 
In the case of the test, it was 45$^\circ$, this being the direction of maximum stress. 

\end{document}