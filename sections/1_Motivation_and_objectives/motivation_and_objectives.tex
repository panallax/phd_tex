\documentclass[../../main.tex]{subfiles}

\begin{document}

Ambition is defined as a strong desire to achieve something, and curiosity is defined as a fervent desire to know or learn. 
Both adjectives have defined and accompanied human evolution throughout history and have allowed us to develop societies to the point where we have even been able to take human beings beyond the planet they have inhabited for hundreds of thousands of years or replace vital organs in living patients. 
Like all species, human beings are capable of giving life. 
But we are the only species that has learned how to give it anew after birth.

Since time immemorial, humans have learned the importance of healing wounds for survival. 
They learned that certain wounds are regenerated by the body autonomously. 
Their ambition and curiosity led them to experiment with whether the application of natural concoctions could aid regeneration or even help regenerate larger wounds. 
For many years, homo sapiens experimented with different mixtures of elements to aid regeneration, but they understood that not every wound could be regenerated. 
However, it was ambition that drove Sushruta in the 2nd century BC to reconstruct the nose of a wounded warrior with skin from his forehead. 
The success or failure of this operation is not well documented, but what we do know is that Sushruta wrote the Suśruta Saṃhitā (Suśruta's Compendium), considered to be one of the most important surviving ancient treatises on medicine. 
Years later, it was ambition that once again drove Saints Damian and Cosmas to replace the cancerous leg of the Roman deacon Justinian with the leg of a recently deceased Ethiopian\cite{damaso}. 
And for many years to come, many attempted to regenerate parts of patients' bodies with their own skin. 
But it was not until 1869 that Carl Bunger succeeded in performing the first reported successful graft. 
Bunger managed to reconstruct a person's nose by rebuilding it with flesh from their own thigh, a technique inspired by the one used by Sushruta 2000 years earlier. 
A few years later, curiosity led Theodor Kocher to wonder about the effects of partially removing certain organs from a human being. 
And in 1883, he observed that the total removal of the thyroid gland caused a series of specific symptoms that we now associate with thyroid hormone deficiency. 
Kocher reversed these symptoms by implanting thyroid tissue in these people, thus performing the first organ transplant. 
This milestone earned him the Nobel Prize in 1909 and marked a turning point in medicine: it created the field of regenerative medicine. 
From that moment on, Kocher began a journey that many would follow and take even further. 

Today, organ transplants between patients are commonplace due to the regularity with which they are performed. In 2023 alone, 172,409\footnote{Data extracted from \href{https://www.transplant-observatory.org/}{Global Observatory on Donation and Transplantation}} organ transplants were performed worldwide. 
Hundreds of milestones have been achieved to date, and we are now capable of transplanting all vital organs except the brain and certain digestive organs. 
The greatest limitation on organ transplantation is the low availability of donor organs, which is determined by a medical culture committed to life and social solidarity. 
For this reason, the ambition and curiosity of the scientific community has for years been motivating researchers to consider the possibility of artificially generating organs for use in transplants. 
Thanks to this, we can now manufacture inert parts that mimic and replace bone or teeth and are implanted in patients who need bone or dental replacement. 
But this does not satisfy infinite human ambition, and for years we have dreamed of being able to manufacture living organs with living material. 
Many scientists are currently working on designing methods and tools to make this dream a reality. 
Thanks to the development of 3D printing, this dream is getting closer to becoming a reality every day, but there is still a long way to go and many other paths to create. 
3D printing allows us to create very complex geometries such as those found in the human body, but more significantly, it allows us to use materials containing living cells to print structures that can subsequently create living tissue.
 
Despite this great advance, the work that needs to be done to continue developing this technology is immense, and the need for more people to collaborate in its development is imperative if we are ever to be able to manufacture organs with the patient's own cells that can be implanted on demand. 
Only through ambition and curiosity can this milestone be achieved, which would improve the quality and life expectancy of human beings.

\section{Motivation}
\subfile{1_Motivation/motivation}

\section{Research questions and objectives}
\subfile{2_Objectives/objectives}
\end{document}
