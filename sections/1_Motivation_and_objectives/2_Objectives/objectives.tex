\documentclass[../../../main.tex]{subfiles}
\begin{document}    

The primary aim of this study is to create porous organic scaffolds that can be printed without the need for support in any situation through extrusion, as well as to adjust to a specific load scenario.
This is where the first question that this research aims to address arises:

\begin{center}
   \textit{\textbf{RQ1: How can we develop improved porous structures, suitable for 3D Printing through extrusion, for use in tissue engineering, \\ in a physics-driven way?}}
\end{center}

Furthermore, by leveraging the lightness of porous structures and other features such as heat dissipation or vibration absorption, the question arises as to whether they could have any applications beyond tissue engineering. 
Given that lightness is a fundamental aspect to consider in space missions, this structure could help solve some existing problems or replace parts currently used in space. 
Therefore, in addition to the question raised above, a new question arises:

\begin{center}
   \textit{\textbf{RQ2: Is it possible to generate bio-inspired \\structures that can be used in a real space scenario?}}
\end{center}


Furthermore, conventional 3D printing methods are based on the solidification of molten material on top of already solidified layers. 
Since each layer of molten material is deposited on top of solidified material, the layers of material do not end up bonding homogeneously, but rather the manufactured body ends up being composed of many sheets bonded together. 
This, combined with the rough surface obtained, facilitates the generation of cracks, which makes additively manufactured parts even more fragile, as these aspects allow the crack to propagate easily through the interface between the layers, separating them.
To solve this problem, this project proposes breaking with the manufacturing methodology used by additive manufacturing and not using a layer-by-layer methodology to manufacture the scaffolds. 
Taking advantage of the ease with which the numerical control codes used by 3D printers can be generated, the aim is to use a novel 3D printing methodology and, instead of 3D printing layer by layer, to print in a non-flat manner. 
Avoiding the presence of stacked slices would generate smoother surfaces and micro-structurally homogeneous parts, reducing the fragility added to conventionally 3D-printed parts. 
Thus, in addition to the objective of generating stochastic structures, the objective of manufacturing them in a non-flat manner is added.

For this to be possible, certain considerations must be taken into account that completely change the characterisation of the desired porous structure and the manufacturing process.
Firstly, this approach can only be used in extrusion-based 3D printing methods. 
Secondly, one of the major limitations of non-flat printing is that not all shapes can be printed.
Generally, only those composed of cylindrical elements can be printed. 
This is due to the continuous and directional nature of the extruded material.
Given that the printing method initially proposed in this research is extrusion, and that strut-based structures are a type of open-cell porous structure, these limitations do not affect the feasibility of the main objective.
However, it does imply the need for almost complete control over the structure generation process, as it adds restrictions to the design process.
In non-planar printing, there is a design constraint that does not exist in conventional printing. 
Since the extruder can move in any direction, it must be taken into account that it could cross the already printed material and tear it off. 
Therefore, during the design process, a constraint must be added that all struts maintain a minimum distance from each other to facilitate printing. 
This minimum distance must be equal to the size of the extruder radius.
This means that any design method that does not allow control of the position of the edges must be ruled out. 
Far from being undesirable, this is advantageous, as it forces the use of parametric design methodologies that allow control of other characteristics of the final structure, such as pore size or geometric parameters.

As the ultimate goal of these structures is to be used as bone scaffolds, coexistence with living cells means that no support material should be used. 
To avoid using external support structures, the design must take into account the self-sustainability of the structure, so that no support structures are needed. 
This could be done by controlling the angle of the struts with respect to the direction of construction.
In conventional 3D printing, angles greater than 45$^\circ$ should be printable without support, as at least half of the current printed layer is in contact with the lower one, ensuring adequate adhesion and, therefore, stability.
However, by controlling certain printing parameters, the maximum allowable overhang angle may be less than 45$^\circ$. 
To the author's knowledge, how this affects non-flat printing has not yet been studied.
Therefore, another objective of this research is the study and use of self-supporting structures printed using non-flat printing.
For all these reasons, the final research question that this project aims to answer arises:

\begin{center}
\textit{\textbf{RQ3: Is it possible to 3D-print self-supported structures in a non-planar way?}}
\end{center}

\subsection{Outline}

To facilitate understanding of the work carried out and to guide the reader through the entire process undertaken during the development of this research, this manuscript is divided into chapters that reflect the order of the stages followed by the author during the development of this thesis.

\begin{itemize}
    \item \textbf{Chapter I - Background and Research Objectives:} The first chapter includes the motivation that led to the initiation of this research and the questions that were raised at the end of this work.
    
    \item \textbf{Chapter II - State of the art:} The advances made by the scientific and industrial community in the development of stochastic porous structures are presented, together with the limitations encountered, in order to justify the need for this work.

    \item \textbf{Chapter III - Methodology:} This chapter has been divided into two parts. The first part shows the initial solutions proposed based on suggestions made in the literature. The purpose of this section is to justify why the solutions proposed in the literature for generating self-sustaining stochastic lattice structures by connecting random points are not viable for producing such structures. In addition, this section also aims to show the different stages reached in the development of the methodology that inspired the methodology finally implemented. The second part shows the methodology finally implemented to generate the structures proposed in this work.
    
    \item \textbf{Chapter IV - Results:}  In this section an analysis of the capacity of the developed algorithm and its limitations is conducted in the form of a results analysis.
    
    \item \textbf{Chapter V - Case Studies:} This chapter covers the development of the two case studies that aim to answer research questions RQ2 and RQ3. The first section shows how non-planar printing was achieved through high-temperature extrusion. The second section shows the application of the algorithm developed to degenerate a structure that serves as a thruster bracket for a satellite. Both sections include the results obtained for each of the case studies.

    \item \textbf{Chapter VI - Discussion and conclusion:} This chapter consists of several sections. First, an assessment of the degree of satisfaction of the research questions posed is also provided. Then, a conclusion of all the work carried out is included, and future lines of research within the framework of this investigation are proposed. Finally, the most significant academic and technological contributions made during the course of this research are detailed in the last section.

\end{itemize}

\end{document}