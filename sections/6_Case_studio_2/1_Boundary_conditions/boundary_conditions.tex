\documentclass[../../../main.tex]{subfiles}
\begin{document}
The satellite propulsion system bracket is responsible for connecting the satellite body to the rocket engine, keeping the engine fixed to the satellite and transmitting the thrust generated by the thruster.  
Therefore, it is anchored to both the satellite body and the rocket engine. 
In addition, it may include moving parts that allow the thrust vector of the engine to be modified if it has a fixed nozzle. 
It is considered a critical structure, as any damage to it could compromise the integrity of the mission. 
Therefore, the loads demanded will be high to ensure that the design criteria are met with a safety margin at all parts of the mission. 

During a space mission, two scenarios define the mechanical loads: launch and operation. 
During satellite operation, the propulsion system is designed to ensure orbital correction and maintenance, attitude control, transfer manoeuvres and safe disposal at the end of its useful life.
Therefore, its use will be occasional, and the thrust required will not be very high as the satellite will operate in a zero-gravity environment. 
The greatest stresses during operation will be thermal, as the satellite will be continuously subjected to large temperature gradients due to solar exposure and Earth's shadow. 
Furthermore, as it is in a vacuum, there are no thermal convection phenomena, so heat dissipation only occurs through radiation or conduction through the structure. 
During launch, the satellite is carried inside the rocket and undergoes the same acceleration as the rocket. 
The rocket must overcome gravity to escape it, so the acceleration values reached will be several times the value of gravity. 
This global acceleration occurs in the direction of the main thrust and is transmitted as quasi-static loads to all elements of the rocket. 
In addition, the acoustic field and vibrations transmitted from the launcher generate a spectrum of random accelerations. 
The noise during the first few seconds of launch generates acoustic pressures that reach high decibel levels and are transmitted as structural vibrations. 
Also, during the separation of the launcher stages, high-frequency dynamic loads are generated, although the frequencies reached are lower than those experienced during the first stage of launch. 
As the dynamic loads during launch are greater than those experienced during operation, they will determine the minimum structural requirements that the bracket must satisfy. 

Due to the confidentiality of the satellite design and its components, no actual images or details of the propulsion systems used in the satellite will be included, but a conceptual image of the design space and a summary of the boundary conditions are included. 
During the launch phase, the propulsion system bracket will be subjected to a severe mechanical environment defined by the qualification loads established for the satellite. 
Firstly, it must withstand quasi-static loads of up to 10 \textit{g} in the mounting plane (in-plane) and 13 \textit{g} in the perpendicular direction (out-of-plane), which represent the overall accelerations transmitted by the launch vehicle. 
The bracket will also be exposed to sinusoidal vibrations in the range of 5–110 \textit{Hz}, with displacements of up to ±11 \textit{mm} at low frequencies and increasing levels of acceleration at higher frequencies. 
In addition, it must bracket a random vibration environment in the range of 20–2000 \textit{Hz}, characterised by a power density spectrum with increasing and decreasing slopes depending on the frequency and with global integrated values of up to 15 \textit{g} RMS in the out-of-plane direction. 
These conditions conservatively reproduce the dynamic stresses that the launcher transmits to the satellite and ensure that the bracket can maintain the structural integrity and correct attachment of the propulsion system during ascent. 
The space design for the bracket is shown in \cref{fig:bracket}. 
The bracket will be anchored to the satellite body using four M5 bolts. 
It will also have a hole along the entire structure with a diameter of 13 \textit{cm}, into which the motor will be inserted. 

\begin{figure}[!htbp]
    \centering
    \includegraphics[width= 0.6\textwidth]{imgs/bracket.png}
    \caption{Illustrative representation of the design space for the thruster bracket.}
    \label{fig:bracket}
\end{figure}
\end{document}