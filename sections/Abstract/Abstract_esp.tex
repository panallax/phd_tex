\documentclass[../../main.tex]{subfiles}
\begin{document}    
El desarrollo de la fabricación aditiva ha permitido aumentar la versatilidad de los métodos de fabricación abriendo la puerta a fabricar diseños que serían imposibles mediante otros métodos. Mediante la adición de material capa por capa se pueden fabricar diseños complejos con propiedades mecánicas optimizadas para tareas específicas. Junto a ello, el avance en algoritmos de geometría computacional ha brindado excelentes sinergias a la hora de generar intrincadas geometrías que pueden ser finalmente fabricada. Pero, a pesar de brindar nuevas oportunidades para fabricar diseños más prometedores, esta metodología también trae consigo nuevos retos de diseño. Bajo ciertas condiciones puede ser necesario el uso de material de soporte externo para fabricar algunas piezas. Dicho material es eliminado tras la fabricación, pero debido al diseño de algunas piezas puede ser imposible, o indeseable, eliminar dicho material. Por ello, los diseños se realizarán minimizando el uso de este material. 

Las estructuras porosas aleatorias son un tipo de estructuras que la fabricación aditiva ha vuelto a poner en el punto de mira ya que son muy complejas de fabricar mediante otros métodos y presentan excelentes propiedades mecánicas como excelente ratio rigidez-peso, anisotropía mecánica y absorción de energía. Dichas propiedades son relevantes en campos con aplicaciones de alto requisito mecánico como la biomedicina y la aeronáutica. A pesar de acercar este tipo de estructuras a aplicaciones reales, debido a su aleatoriedad el uso de material de soporte es generalmente inevitable para fabricarlas. Y las metodologías para diseñarlas son reducidas.

Este trabajo propone un método de diseño de estructuras porosas estocásticas que sean fabricables sin material de soporte. Para ello se ha desarrollado un algoritmo de geometría computacional que genera estructuras similares a grafos basado en la tetraedralización controlada de un volumen a partir de teselaciones de Dealunay modificadas. Dicha tetraedralización garantiza que todas las aristas presentes en la estructura cumplan las condiciones geométricas que garantizan la autosustentabilidad de la estructura.

Se ha comprobado la viabilidad de las estructuras propuestas mediante dos casos de estudio. Primero se propone utilizar estas estructuras como andamios que permitan utilizar diseños más similares a los biológicos en bioimpresión ósea hibrida. Además, con el fin de eliminar las capas de impresión y dotar de mayor resistencia al andamio, se propone imprimirlo de forma no planar. Para ello, se diseñó un extrusor metálico calefactado que permitiera imprimir mediante extrusión a alta temperatura. En el segundo caso de estudio se estudia la versatilidad del algoritmo desarrollado en un caso de aplicación espacial. Se propone un nuevo diseñó de adaptador del sistema propulsivo de un satélite de la Agencia Espacial Europea. 

Los resultados obtenidos muestran la viabilidad y versatilidad del algoritmo desarrollado con el que se han podido generar estructuras porosas estocásticas que han sido fabricadas en diversos materiales y mediante diferentes técnicas de fabricación. Además, dichas estructuras han mostrado un mejor comportamiento mecánico que los patrones de relleno más comúnmente utilizados en la literatura como son el Gyroide y Voronoi. Mostrando así que esta metodología presenta un gran potencial de aplicación en campos como la medicina o el sector espacial. También, los avances realizados de la impresión no planar añaden un nuevo escalón en el desarrollo de esta metodología de impresión demostrando que una fabricación sin capas puede llegar a ser factible.
\end{document}