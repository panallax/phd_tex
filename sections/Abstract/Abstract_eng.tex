\documentclass[../../main.tex]{subfiles}
\begin{document}    
The development of additive manufacturing has increased the versatility of manufacturing methods, opening the door to designs that would be impossible using other methods. By adding material layer by layer, complex designs with mechanical properties optimised for specific tasks can be manufactured. In addition, advances in computational geometry algorithms have provided excellent synergies when it comes to generating intricate geometries that can ultimately be manufactured. However, despite offering new opportunities to manufacture more promising designs, this methodology also brings with it new design challenges. Under certain conditions, it may be necessary to use external support material to manufacture some parts. This material is removed after manufacturing, but due to the design of some parts, it may be impossible or undesirable to remove it. Therefore, designs will be made to minimise the use of this material. 

Random porous structures are a type of structure that additive manufacturing has brought back into the spotlight, as they are very complex to manufacture using other methods and have excellent mechanical properties such as an excellent stiffness-to-weight ratio, mechanical anisotropy and energy absorption. These properties are relevant in fields with high mechanical requirements, such as biomedicine and aeronautics. Despite bringing this type of structure closer to real applications, due to their randomness, the use of support material is generally unavoidable in their manufacture. Furthermore, the methodologies for designing them are limited.

This work proposes a method for designing stochastic porous structures that can be manufactured without support material. To this end, a computational geometry algorithm has been developed that generates graph-like structures based on the controlled tetrahedralisation of a volume using modified Dealunay tessellations. This tetrahedralisation ensures that all the edges in the structure meet the geometric conditions that guarantee the structure's self-sustainability.

The viability of the proposed structures has been verified through two case studies. First, it is proposed to use these structures as scaffolds that allow the use of designs more similar to biological ones in hybrid bone bioprinting. In addition, in order to eliminate the printing layers and provide greater resistance to the scaffold, it is proposed to print it in a non-planar form. To this end, a heated metal extruder was designed to enable high-temperature extrusion printing. The second case study examines the versatility of the algorithm developed in a space application. A new design for the propulsion system adapter of a European Space Agency satellite is proposed. 

The results obtained show the viability and versatility of the algorithm developed, which has been used to generate stochastic porous structures that have been manufactured in various materials and using different manufacturing techniques. In addition, these structures have shown better mechanical performance than the filling patterns most commonly used in the literature, such as Gyroid and Voronoi. This shows that this methodology has great potential for application in fields such as medicine and the space sector. Furthermore, the advances made in non-planar printing add a new step in the development of this printing methodology, demonstrating that layer-free manufacturing can be feasible.

\end{document}