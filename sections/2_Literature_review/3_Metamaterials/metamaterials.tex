\documentclass[../../../main.tex]{subfiles}
\begin{document}    


In recent years, the concepts of metamaterials and 4D printing have taken on an increasingly important role in the field of bone tissue engineering, offering innovative alternatives to traditional scaffolds \cite{mame202400386, Qin2025}.
Metamaterials are artificial materials whose mechanical and functional response depends not only on the composition of the base material, but mainly on the geometry of their microarchitecture \cite{Qin2025}.
In this context, their application to the design of bone scaffolds has opened the door to implants with advanced properties that seek to overcome the limitations of conventional scaffolds with regular or stochastic porosity \cite{Qin2025, XU2025113829}. 
A paradigmatic example is auxetic metamaterials, characterised by a negative Poisson's ratio. 
When deformed, instead of contracting laterally, they expand, providing a more uniform distribution of loads and greater interfacial contact with the surrounding bone \cite{SHIRZAD2023104905}. 
This can promote both osseointegration and resistance to local collapse under complex loads \cite{adma201301986, C6RA27333E}.
Kim et al. (2021) \cite{Kim2021} studied re-entrant honeycomb and chiral structures as auxetic geometries applicable to bone scaffolds, which, although regular, exhibit superior mechanical behaviour to designs based on simple lattices. 
Also, Zhang et al. (2021)\cite{ZHANG2021107758} designed tubular structures with auxeticity both in the radial direction by through longitudinal repetition and radial revolution of different cross-sections.

Another group of interest are metamaterials based on TPMS, such as the gyroid, diamond or Schwarz-P.
These structures, although based on a regular and periodic base, have a geometric continuity that faithfully reproduces the microarchitecture of trabecular bone, as well as exhibiting an excellent relationship between mechanical strength and permeability \cite{XU2025113829, sciadvaba4261},
These types of metamaterials allow the design of scaffolds with gradients of porosity and stiffness, making them particularly attractive for mimicking the natural transition between cortical and trabecular bone \cite{HUANG2025101664}.
Stochastic controlled approaches have also been developed, where periodic metamaterials are modified by introducing random perturbations in nodes or connections, thus generating hybrid structures that combine the mechanical predictability of periodic designs with the biological variability of stochastic ones \cite{BHUWAL2021103996, HAN2024104199, Pham2019}.

However, the practical implementation of these metamaterials in bone scaffolds poses several challenges.
Firstly, their manufacture requires high-resolution techniques, such as stereolithography (SLA) or selective laser melting (SLM), capable of reproducing complex geometries with interconnected porosity at micrometric scales \cite{JIANG2024110045}. 
Although these technologies enable the creation of prototypes, scalability to larger or mass-produced scaffolds remains limited. 
Furthermore, auxetic metamaterials and TPMS, which rely on highly intricate three-dimensional geometries, pose challenges in terms of support material remnants during printing, particularly in powder or liquid resin fusion techniques \cite{JIANG2024110045}. This requires post-processing that can compromise the sample's integrity or reduce its biocompatibility \cite{C9TB00420C}. 
Another challenge lies in the complexity of modelling and simulating these architectures, as their behaviour cannot be predicted solely by approximations based on apparent density, but requires multiscale models that integrate both the geometric microstructure and the biological environment \cite{Qin2025}.

In parallel, 4D printing has emerged as an extension of 3D printing that adds a temporal dimension to bone scaffolds, giving them the ability to transform or actively adapt to external stimuli such as changes in temperature, humidity, pH or the presence of biomolecules \cite{jmmp9080285, ARIF2022e00203}. 
In this context, scaffolds capable of expanding after implantation have been explored, filling irregular bone defects more efficiently \cite{MIAO2017577}, or even structures programmed to gradually release growth factors in response to physiological stimuli \cite{jmmp9080285}. 
The advantages of this technology lie in the possibility of designing dynamic implants that are not mere passive substitutes, but systems capable of interacting and evolving with the tissue during the regeneration process \cite{mame202400386}. 
However, 4D printing in the bone field is still in very preliminary stages, facing limitations related to the biocompatibility of available smart materials, their long-term mechanical stability, and the need for precise control of the response to biological stimuli \cite{jmmp9080285, ARIF2022e00203, MIAO2017577, RAEI2025e00428}. 
Taken together, both metamaterials and 4D printing represent a clear conceptual break from traditional scaffolds: instead of passive structures with porosity designed to promote osseointegration, an approach is proposed in which the active and programmable geometry of the material is the true vector of functionality. 
Although they have not yet reached a sufficient degree of maturity to replace traditional polymer- or bioceramic-based scaffolds, their ability to offer adjustable mechanical properties, better functional integration and potential adaptability makes them one of the most promising lines in the next generation of bone implants.

\end{document}