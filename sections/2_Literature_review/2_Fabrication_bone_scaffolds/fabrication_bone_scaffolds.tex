\documentclass[../../../main.tex]{subfiles}
\begin{document}  
The additive manufacturing methods used in the literature for printing porous structures vary greatly.
Depending on the usefulness of the manufactured part, the printing methods can be very different.
In general, extrusion predominates for proof-of-concept parts due to its versatility and low cost. 
However, in the manufacture of useful parts, methods based on metal powder are the most widely used due to the high performance offered by metallic materials \cite{polym16142027}.
When it comes to the design of bone scaffolds, the method chosen plays a fundamental role, as it determines not only the geometric fidelity of the scaffold but also its biocompatibility and clinical performance.
In general terms, the technologies are divided into conventional additive manufacturing and bioprinting. 
Among the former, selective laser melting (SLM/SLS) and stereolithography (SLA/DLP) stand out, widely used to produce metal or polymer scaffolds with high geometric precision \cite{KANWAR2021e00167}.
These technologies offer good resolution and reproducibility, but their application in biological environments is limited by the need to use materials that are not always compatible with living cells and by the high costs involved \cite{polym16142027}.
The fabrication of bone scaffolds has undergone considerable evolution in recent years, driven by the parallel development of additive manufacturing and biofabrication technologies. 
Conventional 3D printing techniques, such as FDM, SLS, or SLA, have proven highly effective for producing structural scaffolds with controlled porosity and mechanical strength \cite{KANWAR2021e00167}.
However, when the aim is not only to provide mechanical support but also to integrate living cells and biological components directly during manufacturing, the field of bioprinting becomes essential \cite{HARLEY2021e00147}. 
Biofabrication strategies for bone scaffolds can be broadly divided into three categories: polymer-based fabrication, ceramic or composite printing, and cell-laden bioinks \cite{bioengineering10070759}.

One of the most widely used methods is extrusion-based bioprinting, where viscous bio-inks composed of hydrogels, polymers, and living cells are deposited layer by layer \cite{polym16142027}. 
This approach enables high cell densities and the incorporation of osteogenic factors into the scaffold matrix. 
For example, Daly et al. (2016) \cite{Daly2016} demonstrated hybrid bio-printed scaffolds combining alginate-based hydrogels with thermoplastic polymers, creating composite constructs with both structural and biological functionality.
More recently, Beheshtizadeh et al. (2022) \cite{BEHESHTIZADEH202269} reported extrusion-based printing of biphasic calcium phosphate scaffolds, showing enhanced osteogenesis and regeneration. 
The main limitation of extrusion is its relatively low resolution and the difficulty of balancing mechanical stability versus cell viability, as higher viscosities are required to ensure shape fidelity but can compromise nutrient diffusion and cell survival \cite{gels11080659}.

Another relevant strategy is inkjet bioprinting, where droplets of low-viscosity bio-inks are precisely deposited. 
Its gentleness makes it compatible with fragile cells, and it allows multi-material printing at high speed. Boland et al. (2006) \cite{biot200600081} were among the first to apply inkjet printing to bone tissue engineering, demonstrating spatial control of osteoblasts. 
More recent works have combined inkjet systems with calcium BMP-2-laden droplets to induce osteogenic differentiation \cite{ma18102234}, and advanced inkjet head designs have made it possible to pattern droplets on demand in a three-dimensional structure during printing \cite{Takagi2019}.
However, the low viscosity requirement limits the range of materials and reduces the mechanical strength of the printed scaffolds, making them more suitable for coatings or cell patterning on previously fabricated porous structures \cite{Vanderburgh2017}.

Laser-assisted bioprinting (LAB) has emerged as a high-resolution alternative. In this technique, pulsed lasers generate microdroplets of bio-ink that are transferred to a substrate with great spatial control. 
Guillotin et al. (2010) \cite{Guillotin2010} pioneered LAB for cell patterning, achieving high viability and positioning accuracy. 
In the field of bone printing, this technology is used to position cells in bone defects to promote their generation, but not to manufacture bone scaffolds \cite{Keriquel2017}.
Although still limited by its high cost and technical complexity, LAB offers exceptional precision and viability, making it promising for future scaffold customisation at the cellular scale \cite{Maresca2023}.

In addition to these cell-compatible bioprinting methods, hybrid strategies are increasingly gaining attention. 
These approaches combine structural 3D printing of mechanically robust frameworks with bio-inks deposited afterwards. 
For example, Kang et al. (2016)\cite{Kang2016} developed an integrated tissue-organ printer (ITOP) capable of depositing synthetic polymers for structural stability alongside cell-laden hydrogels for biological function. 
Also, Aisenbrey et al. (2018) \cite{Aisenbrey2018} demonstrated a two-step approach where ceramic-based scaffolds printed by stereolithography were subsequently infused with stem-cell-laden hydrogels, enabling independent design of the support structure and the hydrogel.
Such hybrid methods highlight the potential of supportless, self-sustaining geometries that facilitate cell seeding and vascularisation without compromising printability.

Recent studies have investigated non-planar extrusion as a method for producing more anatomically accurate porous structures with improved surface quality.
For example, Benjamin J. Albert et al. (2022) \cite{ALBERT2022e00242} developed a non-planar 3D print slicer that allows for curved and non-planar trajectories with bioinks inside a bath support, revealing that non-planar manufacturing can produce mechanically tunable properties with homogeneous biomaterials.
Cuan-Urquizo et al. (2018) \cite{RPJ1120180286} and Pérez-Castillo et al. (2023) \cite{Perez-Castillo2023} studied the benefits and limitations of FDM-printed lattice structures on a curved base.
Li et al. (2024) \cite{LI2024112860} printed non-planar structures layer by layer without support. Using a six-axis print head, they were able to rotate the print and print without support, changing the orientation of each layer.
In the field of microfibrillation, Saidy et al. (2020) \cite{Saidy2020} adapted melt electrowriting (MEW) to produce fibrous scaffolds on mandrels or cores with complex anatomical geometries, allowing the orientation of fibres on curved surfaces to be controlled and more realistic microarchitectures to be obtained. 
Kainz et al. (2024) \cite{IJB01535} developed a five-axis inkjet-extrusion printer designed to combine soft and hard phases and print on free-form surfaces, demonstrating the possibility of printing hybrids on unconventional geometries.
All these studies reveal clear advantages—better adaptation to anatomical geometries, reduction of the ‘staircase’ effect, possibility of orienting deposition along load lines—but also common challenges: trajectory planning and toolpath control on curved surfaces, calibration and collision avoidance, different rheological behaviour of bioinks on inclines, and the greater complexity/costs of hardware and post-processing.
Overall, the literature shows that non-planar extrusion is a technically viable and promising route for advanced porous structures, and that its application to customised bone scaffolds will likely grow as the challenges of process control and reproducibility are resolved.

\end{document}